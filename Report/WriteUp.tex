\documentclass[]{article}

\usepackage{listings}
\usepackage{mathtools}
\usepackage{multirow}

\title{Introduction to Intelligent Systems \\ Project1: Rolling Die Mazes}
\author{Joseph Fuchs, Damien Cremilleux}

\begin{document}
\maketitle

\section{Problem definition}
The objective of this project is to solve rolling-die mazes.
A  die start from a location on the maze, and can roll along its edges through a grid, until a goal location is found.
The mazes can contain obstacles, and there are restrictions on which numbers may face up on the die.
The die will always start with the 1 facing up ('visible'), 2 facing up/north, and 3 to the right/east.
Moreover, the number 6 should never be face up on the die, and the number 1 must be on top of the die when the goal location is reached.

\medskip

An example of maze is given below (S is the start location, G is the goal location and * are obstacles).

\lstinputlisting[]{../puzzles/puzzle4.txt}

\medskip

This problem can be formulated as follow:
\begin{itemize}
\item States: The state is determined by both the die position and its orientation.
So, a states description specifies the location (ie the coordinates) and the orientation (which number is facing up, north and east) of the dice on the maze.
\item Initial state: The die is on the start location, with the 1 facing up.
\item Actions: Rolling north, south, east or west. Different subsets of these are possible depending on the orientation of the dice (the number 6 should never be face up).
\item Transitions model: Given a state and action, this return the resulting state.
\item Goal test: The die is on the goal location, with the number 1 on the top.
\item Path cost: Each step costs 1, so the path cost is the number of steps in the path.
\end{itemize}

\section{Heuristics}
We use A* algorithm (discussed in the class) to solve this search problem.
A* includes as a component of its evaluation function a heuristic function, $h(n)$.
In the following sections, we describe the three heuristics chosen to solve this search problem.

\subsection{Uniform-cost search}
For the first heuristic, $h_1$, we chose a very simple solution: $h(n) = 0$.
This means that the evaluation function is now $f(n) = g(n)$ where $g(n)$ is the cost to reach the node.
With this heuristic, we have an uniform-cost search algorithm.
Uniform-cost search first visits the node with the shortest path costs from the root node.

This heuristic is admissible.
It will never overestimate the cost to reach the goal because the heuristic is equal to 0, and we are dealing with positive step costs.

Moreover this heuristic is consistent, which is a stricter requirement than admissibility.
According to the textbook, a heuristic is consistent if, for every node $n$ and every successor $n’$ of n generated by any action $a$, the estimated cost of reaching the goal from $n$ is no greater than the step cost of getting to $n’$ plus the estimated cost of reaching the goal from $n’$:

$
h(n) \leq c(n,a,n’) + h(n,)

0 \leq c(n,a,n’)
$

This is true because we are dealing with positive step costs.

\subsection{Manhattan distance without obstacles}
The second heuristic, $h_2$, is based on the city block distance, or Manhattan distance, which is the sum of the absolute differences of Cartesian coordinates.
To find this heuristic, we take the original problem with fewer restrictions: no obstacles, the die can roll in any direction. 
We illustrate this on the following maze.
The die is on the cell marked ‘D’.
In this case, the heuristic will be $h(n=’D’) = 4$.

\begin{lstlisting}
S . . . .
. D * * *
. * . . G
. . . . .
* . . . .
\end{lstlisting}

So this is a relaxed problem of the first one.
We know that the cost of an optimal solution to a relaxed problem is an admissible heuristic for the original problem.
Moreover, this heuristic is also consistent, because it is the exact cost for the relaxed problem so it must obey the triangle inequality.
Finally, this heuristic gives bigger values than h1. This means that we can expect better results with this heuristics, compare to the previous one. 


\subsection{Manhattan distance with obstacles}
For the last heuristic, $h_3$, we also generate an heuristic from a relaxed problem.
We know only ignore the restriction about the number on top of the die.
Contrary to the heuristic two, we take obstacles into consideration.
So the value given by $h_3$ will be at least as big as $h_2$ for all nodes.
So this heuristic is likely to be more effective than the previous one.
We illustrate this heuristic with the following maze.
Now the heuristic will be $h_3(n=’D’) = 8$.

\begin{lstlisting}
S . . . .
. D * * *
. * . . G
. . . . .
* . . . .
\end{lstlisting}

We used a relaxed problem of the original one.
For the same reasons like $h_2$, this heuristic is admissible and consistent.

\section{Performance metrics}
4
For each heuristics, we ran the five puzzles given in the wording.
The results are in the following tables.

\begin{center}
    \begin{tabular}{| c | c | c | c | c | c | c | c | c | c | c |}
      \cline{2-10} & \multicolumn{3}{|c|}{Puzzle 1} & \multicolumn{3}{|c|}{Puzzle 2} & \multicolumn{3}{|c|}{Puzzle 3} \\
      \hline
    & $h_1$ & $h_2$ & $h_3$ & $h_1$ & $h_2$& $h_3$ & $h_1$ &  $h_2$ & $h_3$ \\ \hline
    Number of nodes generated & 32 &       &       & 95 &  &  & 2 &  & \\ \hline
    Number of nodes visited   & 24 &       &       & 84 &  &  & 3 &  & \\ \hline
    \end{tabular}
\end{center}

\begin{center}
    \begin{tabular}{| c | c | c | c | c | c | c | c |}
      \cline{2-7} & \multicolumn{3}{|c|}{Puzzle 4} & \multicolumn{3}{|c|}{Puzzle 5} \\
      \hline
    & $h_1$ & $h_2$ & $h_3$ & $h_1$ & $h_2$& $h_3$ \\ \hline
    Number of nodes generated & 161 &       &       & 1271 &  &   \\ \hline
    Number of nodes visited   & 149 &       &       & 1260 &  &  \\ \hline
    \end{tabular}
\end{center}


\input{graph1.tex}

\input{graph2.tex}

\input{graph3.tex}

\input{graph4.tex}

\input{graph5.tex}

\section{Discussions}

For the puzzle 3, we see that the number of nodes generated is less than the number of nodes visited.
The explaination is that the initial node is visited, and not expanded.
nn


\end{document}
